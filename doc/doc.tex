\documentclass[a4paper]{article}
\usepackage{polski}
\usepackage[utf8]{inputenc}
\usepackage[margin=1in]{geometry}
\usepackage{listings}
\usepackage{graphicx}
\usepackage{amsmath}

\title{\textbf{
	Wstęp do algorytmów ewolucyjnych \\
	{\em deterministic samping}} - badanie statystycznej istotności metody reprodukcji - zarys projektu}
\author{\textbf{Albert Sadowski, Krzysztof Wodecki}\\
	Wydział Matematyki i Nauk Informacyjnych\\
	Politechnika Warszawska}

\begin{document}
	\maketitle
	\section{Wstęp}	
	Selekcja ukierunkowuje algorytm ewolucyjny w stronę lepszych rozwiązań. Z jednej strony, w fazie reprodukcji, wybierani są rodzice dla operacji genetycznych, z drugiej, w fazie sukcesji, tworzy się nową populację bazową na podstawie osobników znajdujących się w populacji potomnej  i starej populacji bazowej. {\em Deterministic sampling} jest jedną z metod reprodukcji, którą porównamy z inną metodą reprodukcji - ruletkową reprodukcją proporcjonalną.

		\subsection{Reprodukcja ruletkowa}
	Zgodnie z [1], w reprodukcji proporcjonalnej defuniuje się zmienną losową, określając dla każdego osobnika w populacji $P^t$ prawdopodobieństwo jego reprodukcji według wzoru:

	\begin{equation}
		p_r(X) = \frac{\Phi(X)}{\sum_{Y \epsilon P^t}{\Phi(Y)}}
	\end{equation}

	gdzie $\Phi(X)$ - funkcja przystosowania. Prawdopodobieństwo wylosowania osobnika jest wprost proporcjonalne do jego wartości funkcji przystosowania.
{
	\subsection{Deterministic sampling}
	{\em Deterministic sampling} to także metoda reprodukcji proporcjonalnej ([2]). Wartość funkcji dopasowania każdego osobnika w populacji jest dzielona przez średnią wartość funkcji dopasowania w populacji:
	\begin{equation}
		v_d(X) = \lfloor \frac{\Phi(X)}{\sum_{Y \epsilon P^t}{\Phi(Y)} / \overline{P^t}} \rfloor
	\end{equation}
	
	jedynie część całkowita otrzymanej wartości jest brana pod uwagę. Jeżeli wartość $v_d$ jest równa bądź większa od jedności osobnik jest wybierany do kolejnego pokolenia. Pozostałe wolne miejsca w nowej populacji są wypełniane osobnikami z najwyższą wartością $v_d$.

	W przypadku reprodukcji ruletkowej liczność kolejnego pokolenia nie jest znana. Każdy z osobników wybierany jest bądź odrzucany z pewnym prawdopodobieństwem. W przypadku metody {\em dynamic sampling} liczność pochodnej populacji powinna być określona w sposób deterministyczny, tzn. liczba osobników dla których $v_d \geq 1$ jest nieznana, ale liczba "wolnego miejsca" w kolejnym pokoleniu powinna być określona.

	Niech liczba miejsca w kolejnym pokoleniu będzie określona jako: $s = n_d * \alpha$, gdzie $n_d$ to liczba osobników dla których $v_d \geq 1$, $\alpha > 1$ - stała określająca liczność kolejnego pokolenia. Wartość stałej $\alpha$ zostanie ustalona w sposób empiryczny.

	Inną metodą określenia ilości "wolnego miejsca" może być sposób znany z metody {\em Stochastic Remainder Sampling}([2]), gdzie osobniki dla których $v_d < 1$, wybierane są jak w metodzie ruletkowej.

	\section{Ocena metod reprodukcji}
	Dla pewnego problemu $P$ (np. problem plecakowy) zostaną zaimplementowane algorytmy genetyczne uwzględniające dwie różne metody reprodukcji - metodę reprodukcji ruletkowej oraz metodę dynamic sampling.

	Wyniki uruchomień obydwu algorytmów zostaną porównane przy pomocy narzędzi statystycznych. Istotność różnic średnich wartości przystosowania zostanie zweryfikowana odpowiednimi testami statystycznymi. Zestawione zostaną także uśrednione krzywe zbieżności.

	Chcemy uzyskać odpowiedź na pytanie: czy zastosowanie {\em dynamic sampling} daje statystycznie lepsze wyniki niż zastosowanie reprodukcji ruletkowej.

	\section{Bibliografia}
	\begin{enumerate}
		\item Arabas J., {\em Wykłady z algorytmów ewolucyjnych}, Wydawnictwa Naukowo-Techniczne, Warszawa, 2004.
		\item Andare A.V., Errico de L., Aquino A.L.L., Assis L.P., Barbosa C.H.N.R., {\em Analysis of Selection and Crossover Methods used by Genetic Algorithm-based Heuristic to solve the LSP Allocation Problem in MPLS Networks under Capacity Constraints}, EngOpt 2008 - International Conference on Engineering Optimization, Rio de Janeiro, 2008.
		\item Zuylen van, Anke, {\em Deterministic Samping Algorithms for Network Design}, School of Operations Research and Information Engineering, Cornell University, Ithaca, NY 14853.
	\end{enumerate}
	
\end{document}
